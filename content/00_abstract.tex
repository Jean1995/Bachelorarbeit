\thispagestyle{plain}

\section*{Kurzfassung}
In dieser Arbeit wird der semileptonische Zerfall $\overline{B} \to D l \overline{\nu}_l$ untersucht.
Dabei werden die Formfaktoren $f_+(z)$, $f_0(z)$ des Zerfalles durch Fitten an Gittereichrechnungen der Fermilab und MILC Kollaboration ermittelt.
Hierbei wird als Fitfunktion die \enquote{Simplified Series Expansion} mit verschiedenen Ordnungen der Reihenentwicklung verwendet.
Aus den Formfaktoren wird die differentielle Zerfallsbreite des Zerfalles und hieraus die Observable $R(D)$ zu $\SI{0.288+-0.011}{}
$ bestimmt, wobei sich eine Abweichung von $\SI{2.4}{}
\sigma$ zu den experimentellen Werten von Belle und BaBar ergibt.
Diese mögliche Diskrepanz wird zunächst durch die Anpassung der Formfaktoren im Standardmodell und danach durch die Einführung einer neuen Kopplung im Tauonen-Sektor --- beschrieben durch den Wilsonkoeffizienten $C_{\text{S}1}$ --- behoben.
Es wird der mögliche Parameterbereich von $C_{\text{S}1}$, um eine Übereinstimmung von $R(D)$ mit den experimentellen Werten zu erreichen, angegeben.


\section*{Abstract}
\begin{english}
In this thesis the semileptonic decay $\overline{B} \to D l \overline{\nu}_l$ is examined.
Using lattice calculations from the Fermilab and MILC Collaboration, the relevant form factors $f_+(z)$, $f_0(z)$ are determined by fitting to the given data.
As a fit function, the \enquote{Simplified Series expansion} is used, considering several expansion orders.
The differential decay rate as well as the observable $R(D)$ follow from the resulting form factors.
The calculated result $R(D) = \SI{0.288+-0.011}{}
$ portrays an $\SI{2.4}{}
\sigma$ deviation from the experimental results reported by Belle and BaBar.
This possible disagreement is treated by altering the form factors within the Standard Model as well as introducing a BSM coupling, affecting the tau lepton decay rate only, which is described by the Wilson Coefficient $C_{\text{S}1}$.
The region for $C_{\text{S}1}$ to conform with the experimental data of $R(D)$ is presented.
\end{english}
