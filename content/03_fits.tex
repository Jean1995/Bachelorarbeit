\chapter{Fitten der Formfaktoren innerhalb und außerhalb des Standardmodelles}\label{make}
Das Paper \cite{PhysRevD.92.034506} stellt Theoriewerte für die Formfaktoren $f_+(w)$ und $f_0(w)$ für verschiedene Impulsüberträge $w$ zur Verfügung, welche aus Gittereichrechnungen für den exklusiven Zerfall $\overline{B} \to D l \overline{\nu_l}$ stammen.
Die Berechnungen sind dabei \enquote{unquenched}\todo{Übersetzung für unquenched}, berücksichtigen dementsprechend die Dynamik von involvierten Seequarks.
Die Gittereichrechnungen werden unter Verwendung von 14 verschiedenen Konfigurationen, d.h. Kombinationen von Gitterabständen und Massenverhältnissen von leichten Seequarks und Strange-Seequarks, durchgeführt. \todo{Korrektheit und Vollständigkeit der Sachinhalte?}
In Tabelle \ref{tab:data} sind die dem Paper entnommenen Daten und ihre Fehler, sowie in Abbildung \ref{fig:cor_daten} die Korrelationen der Daten untereinander angegeben.
Alle folgenden Berechnungen werden unter Verwendung dieser Werte durchgeführt.

\begin{figure}
  \centering
  \includegraphics[width=0.8\textwidth]{pycode/cormatrix_daten.pdf}
  \caption{Korrelationsmatrix der Daten der Gittereichrechnungen.}
  \label{fig:cor_daten}
\end{figure}

\begin{table}
  \centering
  \caption{Werte der Formfaktoren aus Gittereichrechnungen für verschiedene Impulsüberträge.}
  \label{tab:data}
  \sisetup{table-format=1.2}
  \begin{tabular}{
    S[table-format=1.2]
    S[table-format=1.4]
    @{${}\pm{}$}
    S[table-format=1.4]
    S[table-format=1.4]
    @{${}\pm{}$}
    S[table-format=1.4]
  }
  \toprule
  {$w$} & \multicolumn{2}{c}{$f_+(w)$} & \multicolumn{2}{c}{$f_0(w)$} \\
  \midrule
  1 & 1.1994 & 0.0095 & 0.9026 & 0.0072 \\
  1.08 & 1.0941 & 0.0104 & 0.8609 & 0.0077 \\
  1.16 & 1.0047 & 0.0123 & 0.8254 & 0.0094 \\
  \bottomrule
  \end{tabular}
\end{table}
\todo{Muss ich hier auch noch mal zusätzlich die Quelle angeben, die ich ja eigentlich schon im Fließtext angegeben habe?}

\section{Methodik des Fittens der Formfaktoren}

Das Ziel ist es, aus den diskreten Werten für einzelne Impulsüberträge in Tabelle \ref{tab:data} die Formfaktoren zu einer kontinuierlichen Größe in $z$ zu erweitern, um beispielsweise aus \eqref{eqn:difzb} die totale Zerfallsbreite ermitteln zu können.
Hierzu wird ein Fit an die gegebenen Daten durchgeführt.

Als Fitfunktion wird eine allgemeine Potenzreihenentwicklung in $z$ der Form
\begin{equation}
  \label{eqn:reihenentwicklung}
  f_i(z) = \frac{1}{P_i(z) \Phi_i(z)} \sum_{k=0}^{N_i} a_{k}^{i} z^{k}
\end{equation}
verwendet mit $i=+$ \todo{Yay or nay?}für $f_+$ und $i=0$ für $f_0$.
Hierbei stellen die $a_{k}^{i}$ die zu bestimmenden Fitparameter und $N_i$ die Ordnung, in der die Formfaktoren jeweils entwickelt werden sollen, dar.
Die Reihenentwicklung in $z$ durchzuführen verbessert die Konvergenz der Funktion, da der Impulsübertrag über die $z$-Parametrisierung, wie in Abbildung \ref{fig:z_kreis} dargestellt, auf $\lvert z \rvert \leq 1$ abgebildet wird.
Somit wird der Einfluss höherer Ordnungen von $z$ auf den Formfaktor verringert. \todo{Formulierung und Sachinhalt}

Die Vorfaktoren der Parametrisierung sind die äußeren Funktionen \todo{Übersetzung?} $\Phi_i(z)$ sowie die Blaschkefaktoren $P_i(z)$, welche der \enquote{Simplified Series Expansion} (SSE) \cite{PhysRevD.79.013008} folgend gewählt werden.
Die Blaschkefaktoren werden hierbei als
\begin{align*}
  P_i(z) = \frac{1}{1 - \frac{q^2(z)}{m_i^2}}
\end{align*}
definiert, um die Resonanzen durch angeregte Zustände in den Formfaktoren zu berücksichtigen. \todo{Das mit den Resonanzen konkretisieren. Anregungszustände?}
Dabei wird als $m_i$ jeweils die niedrigste Resonanz gewählt, da diese den größten Einfluss auf den Formfaktor im kinematisch erlaubten Bereich ausübt.
Außerdem muss beachtet werden, dass die Quantenzahlen der Resonanzen mit den jeweiligen Quantenzahlen der Formfaktoren, $J^P = 0^{+}$ für $f_+$ und $J^P = 1^-$ für $f_0$, übereinstimmen.
Dem Paper \cite{PhysRevD.94.094008} werden die hier verwendeten Massen zu
\begin{align*}
  m_p &= \SI{6.329+-0.003}{\mega\electronvolt}
.\\
  m_0 & = \SI{6.716}{\mega\electronvolt}
.
\end{align*}
entnommen.
Die äußeren Funktionen $\Phi_i(z)$ werden im Rahmen der SSE auf $\num{1}$ gesetzt.

Eine Bedingung an die Parametrisierung der Formfaktoren, welche direkt aus Gleichung \eqref{eqn:constraint} folgt, ist
\begin{align*}
  f_+(z_\text{max}) = f_0(z_\text{max})
\end{align*}
mit $q^2(z_\text{max}) = 0$.
Eingesetzt in die Reihenentwicklung \eqref{eqn:reihenentwicklung} folgt daraus die Einschränkung
\begin{equation}
  \label{eqn:einschr}
  a_0^+ = P_+(z_\text{max}) \left( \sum_{k=0}^{N_0} a_k^0 \frac{z_{\text{max}}^k}{P_0(z_\text{max})} - \sum_{k=1}^{N_+} a_k^+ \frac{z_{\text{max}}^k}{P_+(z_\text{max})} \right)
\end{equation}
an die Fitparameter, welche die Gesamtheit der Freiheitsgrade des Fits um einen Freiheitsgrad erniedrigt.

Die Bestimmung der Fitparameter erfolgt mithilfe der Methode der kleinsten Quadrate.
Allgemeines Ziel dieser Methode ist es, die Summe der Residuenquadrate
\begin{align*}
  S = \sum_{i=0}^{} \left( y_{i,+}(z_i) - f_{+}(z_i, \vec{a_+}) \right)^2 + \sum_{i=0}^{} \left( y_{i,0}(z_i) - f_{0}(z_i, \vec{a_0}) \right)^2
\end{align*}
durch die passende Wahl der Fitparameter $\vec{a_+}$ und $\vec{a_0}$ zu minimieren.
Hierbei stellt $y_i$ die Daten der Gittereichtheorie aus Tabelle \ref{tab:data} und $f(z_i)$ die dazugehörige Vorhersage durch die Fitfunktion dar.

Um die Korellationen der Daten untereinander zu berücksichtigen, wird statt der normalen Summe der Residuenquadrate die gewichtete Summe
\begin{align*}
  S = \vec{r}^T \symbf{W} \vec{r},
\end{align*}
minimiert, wobei $\vec{r}$ als Vektorkomponenten die einzelnen Residuen $y_i - f(z_i)$ beinhaltet und $\symbf{W}$ die Gewichtsmatrix ist.
%Als Gewichtsmatrix wird hier eine Diagonalmatrix gewählt, dessen Diagonalelemente jeweils die Inversen der Varianzen von $y_i$ sind.
Als Gewichtmatrix wird die Inverse der Kovarianzmatrix $\symbf{V}$ der Daten gewählt, so dass die Residuen mit der Inverse der Varianz der jeweiligen Theoriewerte gewichtet werden. \todo{Soll hier noch mehr ergänzt werden zur Methodik? Designmatrix, etc.?}

Um ein Konfidenzintervall für die gefittete Funktion zu bestimmen, werden $\SI{1000}{}
$ verschiedene Konfigurationen der fehlerbehafteten Eingangsdaten sowie der fehlerbehafteten Fitparameter zufallsgeneriert, wobei alle Daten als normalverteilt angenommen werden. \todo{Auch hier: ausführlicher?}

%\subsection{Fit der Formfaktoren für \texorpdfstring{$N_+ = \num{2}$, $N_0 = \num{1}$}{N+ = 2, N0 = 1}.}
\section{Ergebnisse der Fits für verschiedene Parameterkombinationen}
\label{sec:fits}

Die Fitparameter werden für verschiedene Kombinationen der Ordnungen $N_+$ der Reihenentwicklung von $f_+$ sowie $N_0$ der Reihenentwicklung von $f_0$ berechnet.
Ein Maß für die Güte des Fits stellt die Anpassungsgüte \todo{Begriff richtig?}
\begin{equation}
  \label{eqn:apg}
  \chi^2 = \sum_{i=0}^{} \left( \frac{y_{i,+}(z_i) - f_{+}(z_i, \vec{a_+})}{\sigma_{y_{i,+}}} \right)^2 + \sum_{i=0}^{} \left( \frac{y_{i,0}(z_i) - f_{0}(z_i, \vec{a_0})}{\sigma_{y_{i,0}}}  \right)^2
\end{equation}
dar, welche jeweils durch die Anzahl der Freiheitgrade $df$ dividiert wird.
Im vorliegenden Fall gilt
\begin{align*}
  df = 6 - (N_1 + N_2 + 1),
\end{align*}
bei sechs vorhandenen Theoriewerten sowie der Nutzung der Nebenbedingung \eqref{eqn:einschr}.
Die berechneten Fitparameter für vier unterschiedliche Kombinationen von $N_+$ und $N_0$ sowie die jeweilige Anpassungsgüte des Fits sind in Tabelle \ref{tab:fitparams} angegeben.
\begin{table}
  \centering
  \begin{subtable}{0.48\textwidth}
    \centering
    
    \sisetup{parse-numbers=false}
    \begin{tabular}{
  S[table-format=-1.0]
	S[table-format=-1.5]
	@{${}\pm{}$}
	S[table-format=-1.5]
	}
	\toprule
  \multicolumn{3}{c}{$N_1 = \num{3} \quad N_2 = \num{2}$} \\
  \midrule
  \rule{0pt}{2.2ex}
  a^+_0: &
 0.851 & 0.007 \\

  \rule{0pt}{2.2ex}
  a^+_1: &
 -3.370 & 0.006 \\

  \rule{0pt}{2.2ex}
  a_\text{+,2}: &
 7 & 3 \\

  \rule{0pt}{2.2ex}
  a_\text{0,0}: &
 0.66872 & 0.00003 \\

  \rule{0pt}{2.2ex}
  a^0_1: &
 -0.13 & 0.03 \\

  \midrule
  \multicolumn{3}{c}{$\chi^2 \,/\, df= \SI{0.029}{}
 \,/\, \num{4}$}		\\
    \bottomrule
    \end{tabular}

  \end{subtable}
  \begin{subtable}{0.48\textwidth}
    \centering
    
    \sisetup{parse-numbers=false}
    \begin{tabular}{
  S[table-format=-1.0]
	S[table-format=-1.5]
	@{${}\pm{}$}
	S[table-format=-1.5]
	}
	\toprule
  \multicolumn{3}{c}{$N_1 = \num{2} \quad N_2 = \num{3}$} \\
  \midrule
  \rule{0pt}{2.2ex}
  a^+_0: &
 0.851 & 0.007 \\

  \rule{0pt}{2.2ex}
  a^+_1: &
 -3.290 & 0.007 \\

  \rule{0pt}{2.2ex}
  a_\text{0,0}: &
 0.66888 & 0.00003 \\

  \rule{0pt}{2.2ex}
  a^0_1: &
 0.1 & 0.2 \\

  \rule{0pt}{2.2ex}
  a_\text{0,2}: &
 -6 & 2 \\

  \midrule
  \multicolumn{3}{c}{$\chi^2 \,/\, df= \SI{1.075}{}
 \,/\, \num{4}$}		\\
    \bottomrule
    \end{tabular}

  \end{subtable}
  \begin{subtable}[t]{0.48\textwidth}
    \vspace{15px}
    \centering
    
    \sisetup{parse-numbers=false}
    \begin{tabular}{
  S[table-format=-1.0]
	S[table-format=-1.5]
	@{${}\pm{}$}
	S[table-format=-1.5]
	}
	\toprule
  \multicolumn{3}{c}{$N_+ = \num{1} \quad N_0 = \num{1}$} \\
  \midrule
  \rule{0pt}{2.2ex}
  a^+_0: &
 0.850 & 0.007 \\

  \rule{0pt}{2.2ex}
  a^+_1: &
 -2.716 & 0.006 \\

  \rule{0pt}{2.2ex}
  a_\text{0,0}: &
 0.670 & 0.005 \\

  \rule{0pt}{2.2ex}
  a^0_1: &
 0.13 & 0.02 \\

  \midrule
  \multicolumn{3}{c}{$\chi^2 \,/\, df= \SI{15.386}{}
 \,/\, \num{3}$}		\\
    \bottomrule
    \end{tabular}

  \end{subtable}
  \begin{subtable}[t]{0.48\textwidth}
    \vspace{15px}
    \centering
    
    \sisetup{parse-numbers=false}
    \begin{tabular}{
  S
	S[table-format=-1.3]
	@{${}\pm{}$}
	S[table-format=-1.3]
	}
	\toprule
  \multicolumn{3}{c}{$N_1 = \num{3} \quad N_2 = \num{3}$} \\
  \midrule
  \rule{0pt}{2ex}
  a^+_0: &
 0.851 & 0.007 \\

  \rule{0pt}{2ex}
  a^+_1 &
 -3.402 & 0.006 \\

  \rule{0pt}{2ex}
  a^+_2 &
 17 & 11 \\

  \rule{0pt}{2ex}
  a_\text{0,0}: &
 0.670 & 0.005 \\

  \rule{0pt}{2ex}
  a^0_1 &
 -0.2 & 0.2 \\

  \rule{0pt}{2ex}
  a_\text{0,2}: &
 11 & 12 \\

  \midrule
  \multicolumn{3}{c}{$\chi^2 \,/\, df= \SI{0.001}{}
 \,/\, \num{1}$}		\\
    \bottomrule
    \end{tabular}

  \end{subtable}
  \vspace{10px}
  \caption{Fitparameter und Anpassungsgüte nach Gleichung \eqref{eqn:apg} für verschiedene Ordnungen der Reihenentwicklung $N_1$ und $N_2$.}
  \label{tab:fitparams}
\end{table}
Zusätzlich sind die Korrelationen der Fitparameter im Anhang in den Abbildungen \ref{fig:fitcor22} bis \ref{fig:fitcor33} angegeben.
Die sich daraus folgenden Fitfunktionen sind zusammen mit den Theoriedaten in den Abbildung \ref{fig:fit22} bis \ref{fig:fit33} dargestellt.
Um die Fitfunktion ist jeweils die $\num{1}$-$\sigma$-Umgebung eingezeichnet.
\begin{figure}
  \centering
  \includegraphics[width=0.8\textwidth]{pycode/plot_22.pdf}
  \caption{Fit an die Theoriewerte für $N_0 = \num{2}$ und $N_+ = \num{2}$.}
  \label{fig:fit22}
\end{figure}
\begin{figure}
  \centering
  \includegraphics[width=0.8\textwidth]{pycode/plot_32.pdf}
  \caption{Fit an die Theoriewerte für $N_0 = \num{3}$ und $N_+ = \num{2}$.}
  \label{fig:fit32}
\end{figure}
\begin{figure}
  \centering
  \includegraphics[width=0.8\textwidth]{pycode/plot_23.pdf}
  \caption{Fit an die Theoriewerte für $N_0 = \num{2}$ und $N_+ = \num{3}$.}
  \label{fig:fit23}
\end{figure}
\begin{figure}
  \centering
  \includegraphics[width=0.8\textwidth]{pycode/plot_33.pdf}
  \caption{Fit an die Theoriewerte für $N_0 = \num{3}$ und $N_+ = \num{3}$.}
  \label{fig:fit33}
\end{figure}

\section{Berechnung der differentiellen Zerfallsbreite sowie \texorpdfstring{$R(D)$}{R(D)}.}

Durch Einsetzen der ermittelten Fitfunktion in Gleichung \eqref{eqn:difzb} lässt sich die differentielle Zerfallsbreite für Elektronen, Myonen sowie Tauonen bestimmen.
Die berechneten Funktionen sind in den Abbildungen \ref{fig:difwq22} bis \ref{fig:difwq33} dargestellt, wobei jeweils zusätzlich das $\num{1}$-$\sigma$-Intervall angegeben ist.
Bei den berechneten Zerfallsbreiten ist zu beachten, dass sich zwischen den Funktionen für $l = e$ und $l = \mu$ vernachlässigbar kleine Unterschiede zeigen, während sich die Funktion für $l = \tau$ deutlich unterscheidet.
Dies liegt an der deutlich größeren Masse des Tauons $m_{\tau}$ im Vergleich zur Elektronenmasse $m_{e}$ oder der Myonenmasse $m_{\mu}$.

\begin{figure}
  \centering
  \includegraphics[width=0.7\textwidth]{pycode/plot_diff_wq22.pdf}
  \caption{Vorhersage der differentiellen Zerfallsbreite für $N_0 = \num{2}$ und $N_+ = \num{2}$.}
  \label{fig:difwq22}
\end{figure}
\begin{figure}
  \centering
  \includegraphics[width=0.7\textwidth]{pycode/plot_diff_wq32.pdf}
  \caption{Vorhersage der differentiellen Zerfallsbreite für $N_0 = \num{3}$ und $N_+ = \num{2}$.}
  \label{fig:difwq32}
\end{figure}
\begin{figure}
  \centering
  \includegraphics[width=0.7\textwidth]{pycode/plot_diff_wq23.pdf}
  \caption{Vorhersage der differentiellen Zerfallsbreite für $N_0 = \num{2}$ und $N_+ = \num{3}$.}
  \label{fig:difwq23}
\end{figure}
\begin{figure}
  \centering
  \includegraphics[width=0.7\textwidth]{pycode/plot_diff_wq33.pdf}
  \caption{Vorhersage der differentiellen Zerfallsbreite für $N_0 = \num{3}$ und $N_+ = \num{3}$.}
  \label{fig:difwq33}
\end{figure}

\section{Betrachtung der Formfaktoren außerhalb des Standardmodelles}
% Schreibe hier erst, warum BSM benötigt wird, und mache dann subsections mit den einzelnen Bereichen von BSM
