\chapter{Fitten der Formfaktoren innerhalb und außerhalb des Standardmodelles}\label{make}

Aus dem Paper \cite{PhysRevD.92.034506} werden Theoriewerte für die Formfaktoren $f_+(w)$ und $f_0(w)$ für verschiedene Impulsüberträge entnommen, welche im Folgenden für alle Fits verwendet werden.
Die Daten stammen aus Berechnungen, welche unter Nutzung der Gittereichtheorie durchgeführt wurden.
In Tabelle \ref{tab:data} sind diese Daten angegeben sowie in Abbildung \ref{fig:cor_daten} die Korrelationen der Daten untereinander.
\begin{table}
  \centering
  \caption{Gittereichrechnungen für verschiedene Impulsüberträge.}
  \label{tab:data}
  \sisetup{table-format=1.2}
  \begin{tabular}{
    S[table-format=1.2]
    S[table-format=1.4]
    @{${}\pm{}$}
    S[table-format=1.4]
    S[table-format=1.4]
    @{${}\pm{}$}
    S[table-format=1.4]
  }
  \toprule
  {$w$} & \multicolumn{2}{c}{$f_+(w)$} & \multicolumn{2}{c}{$f_0(w)$} \\
  \midrule
  1 & 1.1994 & 0.0095 & 0.9026 & 0.0072 \\
  1.08 & 1.0941 & 0.0104 & 0.8609 & 0.0077 \\
  1.16 & 1.0047 & 0.0123 & 0.8254 & 0.0094 \\
  \bottomrule
  \end{tabular}
\end{table}

\begin{figure}
  \centering
  \includegraphics[width=0.9\textwidth]{pycode/cormatrix_daten.pdf}
  \caption{Korrelationsmatrix der Daten der Gittereichrechnungen.}
  \label{fig:cor_daten}
\end{figure}
