\chapter{Anhang}

\nocite{tikzfeynman}

\begin{tikzpicture}
  \begin{feynman}
    \vertex (a1) {\(b\)};
    \vertex[right=1cm of a1] (a2);
    \vertex[right=1cm of a2] (a3);
    %\vertex[right=1cm of a3] (a4) {\(b\)};
    \vertex[right=1cm of a3, label=125:\(V_{cb}\)] (a5);
    \vertex[right=2cm of a5] (a6) {\(c\)};

    \vertex[below=2em of a1] (b1) {\(\overline q\)};
    \vertex[right=1cm of b1] (b2);
    \vertex[right=1cm of b2] (b3);
    %\vertex[right=1cm of b3] (b4) {\(\overline d\)};
    \vertex[below=2em of a6] (b5) {\(\overline q\)};

    \vertex[above=of a6] (c1) {\(\overline \nu_l\)};
    \vertex[above=2em of c1] (c3) {\(l\)};
    \vertex at ($(c1)!0.5!(c3) - (1cm, 0)$) (c2);

    \diagram* {
      {[edges=fermion]
        (b5) -- (b1)
        (a1) -- (a5) -- (a6)
        %(b5) -- (b4) -- (b3) -- (a3) -- (a4) -- (a5) -- (a6),
      },


      (c1) -- [fermion, out=180, in=-45] (c2) -- [fermion, out=45, in=180] (c3),
      (a5) -- [boson, bend left, edge label=\(W^{-}\)] (c2),
    };

    \draw [decoration={brace}, decorate] (b1.south west) -- (a1.north west)
          node [pos=0.5, left] {\(\overline B\)};
    %\draw [decoration={brace}, decorate] (c3.north east) -- (c1.south east)
    %      node [pos=0.5, right] {\(\pi^{-}\)};
    \draw [decoration={brace}, decorate] (a6.north east) -- (b5.south east)
          node [pos=0.5, right] {\(D\)};
  \end{feynman}
\end{tikzpicture}

\begin{figure}
  \centering
  \includegraphics[width=0.7\textwidth]{pycode/plot_w.pdf}
  \caption{Parametrisierung des Impulsübertrages mit $w(q^2)$. Der im vorliegenden Zerfall erlaubte kinematische Bereich nach Gleichung \eqref{eqn:kinematik} wird durch den blauen Kasten verdeutlicht.}
  \label{fig:w_param}
\end{figure}

\begin{figure}
  \centering
  \includegraphics[width=0.7\textwidth]{pycode/plot_z.pdf}
  \caption{Parametrisierung des Impulsübertrages mit $z(q^2)$. Der im vorliegenden Zerfall erlaubte kinematische Bereich nach Gleichung \eqref{eqn:kinematik} wird durch den blauen Kasten verdeutlicht.}
  \label{fig:z_param}
\end{figure}

%zweifach

\begin{figure}
  \centering
  \begin{subfigure}{0.48\textwidth}
    \centering
    \includegraphics[width=\textwidth]{pycode/plot_w.pdf}
    \caption{Parametrisierung mit $w(q^2)$.}
    \label{fig:w_param}
  \end{subfigure}
  \begin{subfigure}{0.48\textwidth}
    \centering
    \includegraphics[width=\textwidth]{pycode/plot_z.pdf}
    \caption{Parametrisierung mit $z(q^2)$.}
    \label{fig:z_param}
  \end{subfigure}
  \caption{Parametrisierung des Impulsübertrages. Der im vorliegenden Zerfall erlaubte kinematische Bereich nach Gleichung \eqref{eqn:kinematik} wird durch den blauen Kasten verdeutlicht.}
\end{figure}
