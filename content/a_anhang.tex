\chapter{Anhang}

\section{Korrelationen der Fitparameter}
Im Folgenden sind die Korrelationsmatrizen für die Fitparameter aus Kapitel \ref{sec:fits} angegeben.
\begin{figure}
  \centering
  \includegraphics[width=0.6\textwidth]{pycode/cormatrix_a_N22.pdf}
  \caption{Korrelationsmatrix der Fitparameter für die Ordnungen $N_+ = \num{2}$ und $N_0 = \num{2}$ der Reihenentwicklung.}
  \label{fig:fitcor22}
\end{figure}

\begin{figure}
  \centering
  \includegraphics[width=0.6\textwidth]{pycode/cormatrix_a_N32.pdf}
  \caption{Korrelationsmatrix der Fitparameter für die Ordnungen $N_+ = \num{3}$ und $N_0 = \num{2}$ der Reihenentwicklung.}
  \label{fig:fitcor32}
\end{figure}

\begin{figure}
  \centering
  \includegraphics[width=0.6\textwidth]{pycode/cormatrix_a_N23.pdf}
  \caption{Korrelationsmatrix der Fitparameter für die Ordnungen $N_+ = \num{2}$ und $N_0 = \num{3}$ der Reihenentwicklung.}
  \label{fig:fitcor23}
\end{figure}

\begin{figure}
  \centering
  \includegraphics[width=0.8\textwidth]{pycode/cormatrix_a_N33.pdf}
  \caption{Korrelationsmatrix der Fitparameter für die Ordnungen $N_+ = \num{3}$ und $N_0 = \num{3}$ der Reihenentwicklung.}
  \label{fig:fitcor33}
\end{figure}

%\section{Andere Sachen}
%\begin{tikzpicture}
%  \begin{feynman}
%    \vertex (a1) {\(b\)};
%    \vertex[right=1cm of a1] (a2);
%    \vertex[right=1cm of a2] (a3);
%    %\vertex[right=1cm of a3] (a4) {\(b\)};
%    \vertex[right=1cm of a3, label=125:\(V_{cb}\)] (a5);
%    \vertex[right=2cm of a5] (a6) {\(c\)};
%
%    \vertex[below=2em of a1] (b1) {\(\overline q\)};
%    \vertex[right=1cm of b1] (b2);
%    \vertex[right=1cm of b2] (b3);
%    %\vertex[right=1cm of b3] (b4) {\(\overline d\)};
%    \vertex[below=2em of a6] (b5) {\(\overline q\)};
%
%    \vertex[above=of a6] (c1) {\(\overline \nu_l\)};
%    \vertex[above=2em of c1] (c3) {\(l\)};
%    \vertex at ($(c1)!0.5!(c3) - (1cm, 0)$) (c2);
%
%    \diagram* {
%      {[edges=fermion]
%        (b5) -- (b1)
%        (a1) -- (a5) -- (a6)
%        %(b5) -- (b4) -- (b3) -- (a3) -- (a4) -- (a5) -- (a6),
%      },
%
%
%      (c1) -- [fermion, out=180, in=-45] (c2) -- [fermion, out=45, in=180] (c3),
%      (a5) -- [boson, bend left, edge label=\(W^{-}\)] (c2),
%    };
%
%    \draw [decoration={brace}, decorate] (b1.south west) -- (a1.north west)
%          node [pos=0.5, left] {\(\overline B\)};
%    %\draw [decoration={brace}, decorate] (c3.north east) -- (c1.south east)
%    %      node [pos=0.5, right] {\(\pi^{-}\)};
%    \draw [decoration={brace}, decorate] (a6.north east) -- (b5.south east)
%          node [pos=0.5, right] {\(D\)};
%  \end{feynman}
%\end{tikzpicture}
%
%\begin{figure}
%  \centering
%  \includegraphics[width=0.7\textwidth]{pycode/plot_w.pdf}
%  \caption{Parametrisierung des Impulsübertrages mit $w(q^2)$. Der im vorliegenden Zerfall erlaubte %kinematische Bereich nach Gleichung \eqref{eqn:kinematik} wird durch den blauen Kasten %verdeutlicht.}
%  \label{fig:w_param_single}
%\end{figure}
%
%\begin{figure}
%  \centering
%  \includegraphics[width=0.7\textwidth]{pycode/plot_z.pdf}
%  \caption{Parametrisierung des Impulsübertrages mit $z(q^2)$. Der im vorliegenden Zerfall erlaubte %kinematische Bereich nach Gleichung \eqref{eqn:kinematik} wird durch den blauen Kasten %verdeutlicht.}
%  \label{fig:z_param_single}
%\end{figure}
%
%%zweifach
%
%\begin{table}
%    \centering
%    \caption{Messdaten Modenbestimmung.}
%    \label{tab:moden}
%    \sisetup{parse-numbers=false}
%    \begin{tabular}{
%	S[table-format=-1.6]
%	@{${}\pm{}$}
%	S[table-format=-1.6]
%	}
%	\toprule
%	\multicolumn{2}{c}{$x$}		\\
%	\midrule
%    0.85    & 0.05    \\
-3.402  & 0.007   \\
18      & 85      \\
0.67004 & 0.00003 \\
-0.20   & 0.03    \\
12      & 92      \\

%    \bottomrule
%    \end{tabular}
%    \end{table}
%
%
%
%% Tabelle Parameter lang
%
%%\begin{table}
%%    \centering
%%    \caption{Messdaten Modenbestimmung.}
%%    \label{tab:moden}
%%    \sisetup{parse-numbers=false}
%%    \begin{tabular}{
%  S
%	S[table-format=-1.6]
%	@{${}\pm{}$}
%	S[table-format=-1.6]
%	}
%	\toprule
%	\multicolumn{3}{c}{$x$}		\\
%	\midrule
%  \rule{0pt}{2ex}
%  a^+_0: &
 0.851 & 0.007 \\

%  \rule{0pt}{2ex}
%  a^+_1 &
 -3.402 & 0.006 \\

%  \rule{0pt}{2ex}
%  a^+_2 &
 17 & 11 \\

%  \rule{0pt}{2ex}
%  a_\text{0,0}: &
 0.670 & 0.005 \\

%  \rule{0pt}{2ex}
%  a^0_1 &
 -0.2 & 0.2 \\

%  \rule{0pt}{2ex}
%  a_\text{0,2}: &
 11 & 12 \\

%
%%    \bottomrule
%%    \end{tabular}
%%\end{table}
%
%\begin{table}
%    \centering
%    \caption{Fitparameter für die Ordnungen $N_+ = \num{3}$ und $N_0 = \num{3}$ der %Reihenentwicklung.}
%    \label{tab:fitparams33}
%    \sisetup{parse-numbers=false}
%    \begin{tabular}{
%  S
%	S[table-format=-1.3]
%	@{${}\pm{}$}
%	S[table-format=-1.3]
%	}
%	\toprule
%  \multicolumn{3}{c}{$N_1 = \num{3} \quad N_2 = \num{3}$} \\
%  \midrule
%  \rule{0pt}{2ex}
%  a^+_0: &
 0.851 & 0.007 \\

%  \rule{0pt}{2ex}
%  a^+_1 &
 -3.402 & 0.006 \\

%  \rule{0pt}{2ex}
%  a^+_2 &
 17 & 11 \\

%  \rule{0pt}{2ex}
%  a_\text{0,0}: &
 0.670 & 0.005 \\

%  \rule{0pt}{2ex}
%  a^0_1 &
 -0.2 & 0.2 \\

%  \rule{0pt}{2ex}
%  a_\text{0,2}: &
 11 & 12 \\

%  \midrule
%  \multicolumn{3}{c}{$\chi^2 \,/\, df= \SI{0.001}{}
 \,/\, \num{2}$}		\\
%    \bottomrule
%    \end{tabular}
%\end{table}
