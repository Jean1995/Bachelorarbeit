\chapter{Theorie des semileptonischen Zerfalles}

Das Feynman-Diagramm in führender Ordnung des untersuchten Zerfalles ist in Abbildung \ref{fig:feynman1} dargestellt.

\todo{Sich für den schönsten Feynmangraphen entscheiden}
\nocite{tikzfeynman}
\begin{figure}
  \centering
  \begin{tikzpicture}
  \begin{feynman}
    \vertex (a1) {\(b\)};
    \vertex[right=1cm of a1] (a2);
    \vertex[right=1cm of a2] (a3);
    %\vertex[right=1cm of a3] (a4) {\(b\)};
    \vertex[right=1cm of a3, label=125:\(V_{cb}\)] (a5);
    \vertex[right=2cm of a5] (a6) {\(c\)};

    \vertex[below=2em of a1] (b1) {\(\overline q\)};
    \vertex[right=1cm of b1] (b2);
    \vertex[right=1cm of b2] (b3);
    %\vertex[right=1cm of b3] (b4) {\(\overline d\)};
    \vertex[below=2em of a6] (b5) {\(\overline q\)};

    \vertex[above=of a6] (c1) {\(\overline \nu_l\)};
    \vertex[above=2em of c1] (c3) {\(l\)};
    \vertex at ($(c1)!0.5!(c3) - (1cm, 0)$) (c2);

    \diagram* {
      {[edges=fermion]
        (b5) -- (b1)
        (a1) -- (a5) -- (a6)
        %(b5) -- (b4) -- (b3) -- (a3) -- (a4) -- (a5) -- (a6),
      },


      (c1) -- [fermion, out=180, in=-45] (c2) -- [fermion, out=45, in=180] (c3),
      (a5) -- [boson, edge label=\(W^{-}\)] (c2),
    };

    \draw [decoration={brace}, decorate] (b1.south west) -- (a1.north west)
          node [pos=0.5, left] {\(\overline B\)};
    %\draw [decoration={brace}, decorate] (c3.north east) -- (c1.south east)
    %      node [pos=0.5, right] {\(\pi^{-}\)};
    \draw [decoration={brace}, decorate] (a6.north east) -- (b5.south east)
          node [pos=0.5, right] {\(D\)};
  \end{feynman}
  \end{tikzpicture}
  \caption{Feynman-Diagramm des semileptonischen Zerfalles $B \to D l \overline \nu_l$ in führender Ordnung.}
  \label{fig:feynman1}
\end{figure}

Im Eingangszustand befindet sich ein $\overline B$-Meson, welches aus einem schweren $b$-Quark und einem leichten Antiquark $\overline q$ ($ \overline q = \overline u, \overline d)$ besteht.
Die schwache Wechselwirkung führt zu einer Umwandlung des beteiligten $b$-Quarks in ein $c$-Quark unter Emission eines $W^{-}$-Bosons. Dieses $W^{-}$-Boson ist ein virtuelles Teilchen und zerfällt direkt weiter in ein Lepton $l$ und das dazugehörige Antineutrino $\overline \nu_l$.
Zusätzlich ist bei diesem Prozess zu beachten, dass die schwache Wechselwirkung nicht an die Masseneigenzustände der Quarks koppelt.
Die korrekten Eigenzustände der schwachen Wechselwirkung ergeben sich durch die Rotation der ursprünglichen Eigenzustände.
Diese Rotation wird durch die unitäre CKM-Matrix beschrieben.
Der für den betrachteten Zerfall relevante Parameter $\lvert V_{cb} \rvert$ beträgt \cite{1703.06124}
\begin{equation}
  \lvert V_{cb} \rvert = \SI{40.49+-0.97e-3}{}
.
\end{equation}

\begin{figure}
  \centering
  \begin{tikzpicture}
  \begin{feynman}
    \vertex (a1) {\(b\)};
    \vertex[right=1cm of a1] (a2);
    \vertex[right=1cm of a2] (a3);
    %\vertex[right=1cm of a3] (a4) {\(b\)};
    \vertex[right=1cm of a3, label=125:\(V_{cb}\)] (a5);
    \vertex[right=2cm of a5] (a6) {\(c\)};

    \vertex[below=4em of a1] (b1) {\(\overline q\)};
    \vertex[right=1cm of b1] (b2);
    \vertex[right=1cm of b2] (b3);
    \vertex[below=2em of a3] (g1);
    \vertex[left=1.8cm of a6] (g2);
    \vertex[right=1cm of b2] (g5);
    \vertex[left=0.5cm of a6] (g6);
    \vertex[right=1.5cm of g1] (g3);
    \vertex[right=0.8cm of g3] (g4);
    %\vertex[right=1cm of b3] (b4) {\(\overline d\)};
    \vertex[below=4em of a6] (b5) {\(\overline q\)};

    \vertex[above=of a6] (c1) {\(\overline \nu_l\)};
    \vertex[above=2em of c1] (c3) {\(l\)};
    \vertex at ($(c1)!0.5!(c3) - (1cm, 0)$) (c2);

    \diagram* {
      {[edges=fermion]
        (b5) -- (b1)
        (a1) -- (a5) -- (a6)
        %(b5) -- (b4) -- (b3) -- (a3) -- (a4) -- (a5) -- (a6),
      },


      (c1) -- [fermion, out=180, in=-45] (c2) -- [fermion, out=45, in=180] (c3),
      (a5) -- [boson, edge label=\(W^{-}\)] (c2),

      (a2) -- [gluon] (g1)
      (b2) -- [gluon] (g1)
      (g1) -- [gluon] (g2)
      (g5) -- [gluon] (g3)
      (g3) -- [fermion, half left] (g4)
      (g4) -- [fermion, half left] (g3)
      (g4) -- [gluon] (g6)

    };

    \draw [decoration={brace}, decorate] (b1.south west) -- (a1.north west)
          node [pos=0.5, left] {\(\overline B\)};
    %\draw [decoration={brace}, decorate] (c3.north east) -- (c1.south east)
    %      node [pos=0.5, right] {\(\pi^{-}\)};
    \draw [decoration={brace}, decorate] (a6.north east) -- (b5.south east)
          node [pos=0.5, right] {\(D\)};
  \end{feynman}
  \end{tikzpicture}
  \caption{Exemplarisches Feynman-Diagramm des semileptonischen Zerfalles $B \to D l \overline \nu_l$ unter Berücksichtigung der starken Wechselwirkung zwischen den beteiligten Quarks.}
  \label{fig:feynman2}
\end{figure}
