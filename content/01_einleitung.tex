\chapter{Einleitung}

Das Standardmodell der Elementarteilchenphysik beschreibt den Aufbau der Materie mit einer beeindruckenden Genauigkeit.
Es führt die Zusammensetzung der Materie auf die Elementarteilchen Quarks, Leptonen, Eichbosonen sowie das Higgs-Boson zurück.
Als Wechselwirkungen existieren die elektromagnetische Wechselwirkung, die starke Wechselwirkung, die schwache Wechselwirkung und die Gravitation, wobei letztere außerhalb des Standardmodells durch die allgemeine Relativitätstheorie beschrieben werden muss.
Aufgrund des großen Erfolges des Standardmodells ist es umso interessanter, Beobachtungen zu finden, welche nicht mit der aktuellen Theorie übereinstimmen oder nicht durch diese beschrieben werden.
Hierzu gehören beispielsweise Neutrinooszillationen oder die Asymmetrie zwischen Materie und Antimaterie im Universum.
All diese Erscheinungen bilden die Grundlage für die Einführung neuer Physik außerhalb des Standardmodells.

Ein Hinweis auf neue Physik durch eine solche Diskrepanz, welche in dieser Arbeit untersucht werden soll, findet sich in semileptonischen Zerfällen von $B$-Mesonen in $D$-Mesonen.
Für diesen Zerfall existieren experimentelle Messungen der Observable $R(D)$, welche jedoch von den theoretischen Berechnungen signifikant abweichen.
Fundamental für die theoretische Beschreibung dieses Zerfalles sind die sogenannten Formfaktoren, welche in dieser Arbeit mit besonderen Blick auf die Observable $R(D)$ untersucht werden sollen.

Zunächst wird in Kapitel \ref{sec:theorie} eine theoretische Beschreibung des zugrunde liegenden Zerfalles sowie in diesem Zusammenhang eine Definition der verwendeten Formfaktoren angegeben.
Zusätzlich werden die Kinematik des Zerfalles und die möglichen Parametrisierungen des auftretenden Impulsübertrages kurz beschrieben.
Anhand von Gittereichrechnungen für diskrete Impulsüberträge aus \cite{PhysRevD.92.034506}, werden daraufhin in Kapitel \ref{make} die Formfaktoren berechnet.
Danach werden sowohl die differentiellen Zerfallsbreiten als auch die Observable $R(D)$ aus den ermittelten Fitergebnissen bestimmt.
Abschließend werden Korrekturen der Formfaktoren sowie mögliche Einflüsse durch neue Physik außerhalb des Standardmodells einbezogen, um eine mögliche Erklärung für die Diskrepanz zu den experimentellen Daten bereitzustellen.
