\chapter{Zusammenfassung und Ausblick}

In dieser Arbeit konnten die Formfaktoren für den semileptonischen Zerfall $\overline{B} \to D l \overline{\nu}_l$ unter Nutzung der \enquote{Simplified Series Expansion} als Fitfunktion aus den vorgegebenen Daten entwickelt werden.
Die Observable $R(D)$ wurde aus den ermittelten Formfaktoren, welche durch die in Tabelle \ref{fig:fit33} angegebenen Fitparametern beschrieben werden, zu
\begin{align*}
  R(D) &= \SI{0.288+-0.011}{}

\end{align*}
bestimmt.
Somit konnte die Abweichung von den experimentellen Daten der Experimente Belle und BaBar um $\SI{2.4}{}
\sigma$ bestätigt werden.
Dies ist auch im Einklang mit dem in Gleichung \eqref{eqn:R_quelle} angegebenen Ergebnis für $R(D)$ aus dem Paper \cite{PhysRevD.92.034506}.
Die hier angewendeten Korrekturen haben exemplarisch gezeigt, wie diese mögliche Diskrepanz mithilfe von neu eingeführten Kopplungen außerhalb des Standardmodells im Tauonen-Sektor quantitativ erklärt werden kann.
Mit diesen Rechnungen kann jedoch nicht ausgeschlossen werden, dass auch Korrekturen der Theorie für Elektronen oder Myonen möglich sind.

Um die berechneten Ergebnisse für die Formfaktoren und somit die Vorhersage für $R(D)$ zu verbessern, wären verbesserte Theoriewerte durch genauere Gittereichrechnungen hilfreich.
Zusätzlich wäre eine größere Anzahl von Theoriewerten für verschiedene Impulsüberträge $q^2$ notwendig, um die Qualität des Fits zu verbessern.
Möglicherweise sind auch Verbesserungen an der Fitmethodik, beispielsweise durch die Wahl einer veränderten Parametrisierung des Impulsübertrages $z(q^2)$, möglich.

Von experimenteller Seite aus sind weitere, genauere Ergebnisse für die Observable $R(D)$ erforderlich, um zu bestätigen oder zu wiederlegen, ob eine Diskrepanz zu den theoretischen Daten tatsächlich vorhanden ist.
Die signifikanten Unterschiede zwischen den Messergebnissen von Belle und BaBar zeigen bereits auf, dass eine abschließende Beurteilung ohne neue experimentelle Daten nicht möglich sein wird.
% Formfaktoren mit der SSE aus den vorgegebenen Daten entwickelt werden
% Im Standardmodell bestätigte Diskrepanz, im Grundatz übereinstimmend mit Ergebnissen des Papers
% Korrekturen zeigen exemplarisch, wie neue Phyik möglcih im Tauonen-Sektor ist und dass die Korrektur der Diskrepanz somtit mögich ist, falls sie bestätigt werden kann
% Bessere Theoriewerte durch genauere Gittereichrechnungen nötig
% Mehr Latticedaten für non-zero-recoil und für weitere Impulsüberträge
% Möglicherweise sind auch Verbesserungen an der Fitmethodik, beispielsweise durch Wahl einer besseren Parametriseirung (z) möglich
% Weitere experimentelle Ergebnisse, um beide Seiten zu konkretisieren
% -> Unterschiede zwischen Belle und BaBar Daten zeigen, dass weitere, genauere experimentelle Messungen von Nöten sind
% -> Auf Tauonen untersucht -> Andere sind mit der Arbeit nicht auszuschließen
% -> Zudem nur exemplarische BSM Physik
